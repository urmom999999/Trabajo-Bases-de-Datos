\documentclass{article}
\usepackage[utf8]{inputenc}
\usepackage{graphicx,float}
\usepackage{hyperref}
\graphicspath{{images/}}
\title{Guia de instalacion de LaTex}
\author{Álvaro Del Valle Fernández}

\begin{document}
\begin{itemize}
    \item {\Large \textbf{Introducción}}
    \begin{itemize}
        \item Introducción a las bases de datos
    \end{itemize}

    \item {\Large \textbf{MongoDB}}
    \begin{itemize}
        \item Historia de MongoDB
        \item Puntos positivos y negativos
        \item Casos de uso
    \end{itemize}

    \item {\Large \textbf{MySQL lite}}
    \begin{itemize}
        \item Historia de MySQL
        \item Puntos positivos y negativos
        \item Casos de uso
    \end{itemize}

    \item {\Large \textbf{PostgreSQL}}
    \begin{itemize}
        \item Historia de PostgreSQL
        \item Puntos positivos y negativos
        \item Casos de uso
    \end{itemize}

    \item {\Large \textbf{Conclusión}}
\end{itemize}
\maketitle

\section{Introducción}
LaTex es un sistema de creación de documentos de alta calidad, utilizado para documentos cientificos, articulos y libros. En esta guia mostrare paso a paso el proceso de instalación de este software.
\section{Instalaccion de LaTex}
El primer paso es instalar el software LaTex en si, los siguientes pasos consistiran en instalar otros elementos para que funcionen correctamente en Visual Studio.
\subsection{Descarga del software}
Para obtener el software debemos acceder a la pagina oficial mediante el link adjuntado:
\url{https://miktex.org/download}
\\Una vez entrada a la pagina debemos hacer click en tu sistema operativo y luego en Download.

\subsection{Instalación del software}
Una vez terminada la descarga ejecutamos el archivo, aceptamos las condiciones, \textit{Next}, \textit{Next}, \textit{Next}, Start y esperamos a que instale. Una vez terminada seleccionamos \textit{Next}, \textit{Next} y Close.
\\Tras esto tendriamos LaTex instalado correctamente en nuestro ordenador.


\section{Instalaccion de Perl}

Necesitamos descargar Perl para su lenguaje, permitiendo numerosas funciones y scripts.
\subsection{Descarga del software}
Para obtener el software debemos acceder a la pagina oficial mediante el link adjuntado:
\url{https://strawberryperl.com/}
\\Una vez en la pagina hacemos click en el ultimo lanzamiento, el archivo tipo MSI.

\subsection{Instalación del software}
Ejecutamos el archivo, \textit{Next}, aceptamos la licencia, \textit{Next},\textit{Next}, Install y Finish.
\\Con esto tendremos este elemento descargado correctamente. Tras este paso se recomienda abrir Perl y buscar actualizaciones, tras ello es recomendable reiniciar el ordenador.

\section{LaTeX Workshop}
Este programa sirve para poder editar correctamente los archivos desde Visual Studio.
\subsection{Descarga del software}
Para obtener el software debemos acceder a la pagina oficial mediante el link adjuntado:
\url{https://marketplace.visualstudio.com/items?itemName=James-Yu.latex-workshop}

\subsection{Instalación del software}
Ejecutamos el programa, le damos a \textit{Next}, aceptar licencia, \textit{Next} e Instalar.
\\Otra opcion mas adecuada es instalarlo directamente desde Visual Studio, accediendo a Extensiones, LaTeX Workshop e Instalar.
\\Con ello lo tendriamos correctamente instalado.
\section{Conclusión}
Tras realizar todos los pasos debemos poder ver el boton de TEX en la barra izquierda de Visual Studio. 

Si seleccionamos "Build LaTex project" nos creará los archivos correctamente, pudiendo crear nuestro primer .tex.


\end{document}