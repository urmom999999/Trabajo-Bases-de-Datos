\documentclass{article}
\usepackage[utf8]{inputenc}
\usepackage{graphicx,float}
\usepackage{hyperref}
\graphicspath{{images/}}
\title{\Large \textbf{Guia de instalacion de LaTex}}
\author{Álvaro Del Valle Fernández}
\begin{document}
\maketitle

\begin{itemize}
    \item {\Large \textbf{Introducción}}

        \item {\Large \textbf{MySQL}}
    \begin{itemize}
        \item Historia de MySQL
        \item Puntos positivos y negativos
        \item Casos de uso
    \end{itemize}

    \item {\Large \textbf{MongoDB}}
    \begin{itemize}
        \item Historia de MongoDB
        \item Puntos positivos y negativos
        \item Casos de uso
    \end{itemize}

    \item {\Large \textbf{SQLite}}
    \begin{itemize}
        \item Historia de SQLite
        \item Puntos positivos y negativos
        \item Casos de uso
    \end{itemize}

    \item {\Large \textbf{PostgreSQL}}
    \begin{itemize}
        \item Historia de PostgreSQL
        \item Puntos positivos y negativos
        \item Casos de uso
    \end{itemize}

    \item {\Large \textbf{Conclusión}}
\end{itemize}
\newpage


\section{Introducción}
En esta práctica veremos tres de los principales sistemas de gestion de gestion de bases de datos junto las diferencias entre ellos,
 veremos tambien su historia y cual se adapta mejor a cada tipo de proyecto segun sus puntos fuertes y sus puntos debiles.
 Elegir la base de datos adecuada para un proyecto es fundamental, adaptandose a las necesidades de este.


\section{MySQL}
\subsection{Historia de MySQL}
En 1995 Michael Widenius junto con su equipo desarrollaron MySQL, siendo lanzado completamente en 2001,
 este consiste en sistema de almacenamiento
 de archivos orientado a uso domestico y profesional, creado para cubrir la demanda de bases de datos de codigo abierto
 y sencillas. En 2010 fue comprado por Oracle. Hoy en dia es utilizada por muchos de los sitios web mas importantes, como Facebook, Twitter, Wikipedia, Youtube, Google...
\subsection{Puntos positivos y negativos}
MySQL es sencillo de aprender, bueno para leer datos, compatible con multiples sistemas operativos,
 cuenta con un buen soporte de la comunidad al ser codigo open-source, perfecto para aplicaciones web
  y para bases de datos de tamaño pequeño y mediano.\\

Sus puntos debiles son su mal rendimiento en procesos complicados, siendo limitado para actualizar y 
ampliar la base de datos.

\subsection{Casos de uso}
MySQL es perfecto para proyectos que necesiten leer gran cantidad de datos rapidamente, no para editado de datos.
 Con su codigo abierto con ayuda de la comunidad, este resulta sencillo de utilizar por lo que es perfecto para 
 proyectos sencillos o complicados.


\newpage
\section{MongoDB}
\subsection{Historia de MongoDB}
MongoDB fue lanzado en el 2009, su objetivo era crear un nuevo sistema de bases de datos para
 gestionar datos no estructurados, usando modelos no SQL y simplificando numerosos elementos.
\subsection{Puntos positivos y negativos}
Perfecto para proyectos que precisen de escalabilidad, siendo automatico, su flexibilidad ayuda a realizar
 los proyectos de forma mucho mas sencilla y rapida. Destaca en escritura.\\
 Sus puntos negativos son problemas de robustez del sistema, en casos de transacciones y posibles duplicados.
  Tambien existen problemas de consumo de RAM, siendo mucho mas alta que las otras opciones. 
  Ciertos elementos pueden resultar mas complejos al ser elementos originalmente de SQL.
\subsection{Casos de uso}
Perfecto para proyectos no estructurados o semi estructurados que puedan necesitar escalabilidad,
 siendo esta automatica y mucho mas sencilla. Facil de realizar proyectos con ella siendo mas rapido que las 
 otras alternativas.



\newpage
\section{SQLite}
\subsection{Historia de SQLite}
Lanzado en el 2000, se centra en ser "lite" ligera, sin necesidad de servidor ni muchos de los elementos pesados 
de las otras bases de datos. Este nuevo sistema fue perfecto para sistemas de bajos recursos como moviles y sistemas 
mas basicos, creando la base de datos como archivo.
\subsection{Puntos positivos y negativos}
SQLite no precisa de servidor y la instalacion es extremadamente simple y muy ligera, la base de datos es solo un archivo siendo
 extremadamente facil de mover. Es muy compatible con casi todas las plataformas y es de dominio publico, olvidandose de licencias.
Problemas con escrituras multiples, saturandose, al no existir un servidor como tal, los usuarios acceden al documento directamente,
 pudiendo generar problemas de escritura. Al ser solo un archivo y estar diseñado para elementos ligeros no se adapta bien a 
 aplicaciones grandes y su escalabilidad es limitada. Sus opciones avanzadas son limitadas comparado con las otras.
 \subsection{Casos de uso}
SQLite destaca por ser extremadamente portable, por lo que el caso de uso es cuando se necesite una base de datos 
muy ligera (menos de un MB) y extremadamente portable, que no consuma muchos recursos y sin multiples usuarios modificando la base de datos.

\newpage
\section{PostgreSQL}
\subsection{Historia de PostgreSQL}
Lanzada en 1996 siendo open source, PostgreSQL  esta basada en la anterior Postgres, siendo esta la evolución. Se centra en crear una 
base de datos relacional orientada a objetos, centrada en estabilidad y opciones muy avanzadas.
\subsection{Puntos positivos y negativos}
PostgreSQL destaca en su consistencia, siendo muy robusto y siguiendo los estandares SQL, cuenta con todas las 
opciones avanzadas centrandose en estabilidad mas que en velocidad.
Sus puntos negativos son su complejidad, al contar con todas las opciones avanzadas y los niveles mas altos de seguridad
es complicada de aprender. Es mas lenta que las otras alternativas debido a la seguridad y puede consumir mucha 
RAM comparada con las mencionadas anteroirmente.
\subsection{Casos de uso}
Perfecto para proyectos que precisen de estabilidad y precisión, como es el caso de bases de datos de bancos y sistemas
de análisis. No se permite la perdida o duplicado de datos en estos entornos por lo que este es uno de los sistemas mas
 estables para ello. Cuando se necesisten movimientos muy complicados y con estabilidad este es el sistema adecuado.

\section{Conclusión}
Tras analizar cada una y conociendo directamente MongoDB y MySQL, estas dos son las que mas me interesan para mis 
futuros proyectos. SQLite es extremadamente interesante por lo que aprenderé mas de ella pero PostgreSQL no lo veo 
tan necesario para mis proyectos, siendo un sistema dificil de aprender y con opciones que no necesito en mi caso.\\
    MongoDB es la que me resultó mas intuitiva, adaptandome bien y solo teniendo problemas con elementos externos 
del servidor.
\end{document}