\documentclass{article}
\usepackage[utf8]{inputenc}
\usepackage{graphicx,float}
\usepackage{hyperref}
\graphicspath{{images/}}
\title{\Large \textbf{Guia de instalacion de LaTex}}
\author{Álvaro Del Valle Fernández}
\begin{document}
\maketitle

\begin{itemize}
    \item {\Large \textbf{Introducción}}


    \item {\Large \textbf{MongoDB}}
    \begin{itemize}
        \item Historia de MongoDB
        \item Puntos positivos y negativos
        \item Casos de uso
    \end{itemize}

    \item {\Large \textbf{MySQL lite}}
    \begin{itemize}
        \item Historia de MySQL
        \item Puntos positivos y negativos
        \item Casos de uso
    \end{itemize}

    \item {\Large \textbf{PostgreSQL}}
    \begin{itemize}
        \item Historia de PostgreSQL
        \item Puntos positivos y negativos
        \item Casos de uso
    \end{itemize}

    \item {\Large \textbf{Conclusión}}
\end{itemize}
\newpage


\section{Introducción}
En esta práctiva veremos tres de los principales sistemas de gestion de gestion de bases de datos junto las diferencias entre ellos,
 veremos tambien su historia y cual se adapta mejor a cada tipo de proyecto segun sus puntos fuertes y sus puntos debiles.


\section{MongoDB}
\subsection{Historia de MongoDB}
\subsection{Puntos positivos y negativos}
\subsection{Casos de uso}


\section{MySQL lite}
\subsection{Historia de MongoDB}
\subsection{Puntos positivos y negativos}
\subsection{Casos de uso}


\section{PostgreSQL}
\subsection{Historia de MongoDB}
\subsection{Puntos positivos y negativos}
\subsection{Casos de uso}


\section{Conclusión}




\end{document}