\documentclass{article}
\usepackage[utf8]{inputenc}
\usepackage{graphicx,float}
\usepackage{hyperref}
\graphicspath{{images/}}
\title{\Large \textbf{Guia de instalacion de LaTex}}
\author{Álvaro Del Valle Fernández}
\begin{document}
\maketitle

\begin{itemize}
    \item {\Large \textbf{Introducción}}

        \item {\Large \textbf{MySQL}}
    \begin{itemize}
        \item Historia de MySQL
        \item Puntos positivos y negativos
        \item Casos de uso
    \end{itemize}

    \item {\Large \textbf{MongoDB}}
    \begin{itemize}
        \item Historia de MongoDB
        \item Puntos positivos y negativos
        \item Casos de uso
    \end{itemize}

    \item {\Large \textbf{SQLite}}
    \begin{itemize}
        \item Historia de SQLite
        \item Puntos positivos y negativos
        \item Casos de uso
    \end{itemize}

    \item {\Large \textbf{PostgreSQL}}
    \begin{itemize}
        \item Historia de PostgreSQL
        \item Puntos positivos y negativos
        \item Casos de uso
    \end{itemize}

    \item {\Large \textbf{Conclusión}}
\end{itemize}
\newpage


\section{Introducción}
En esta práctiva veremos tres de los principales sistemas de gestion de gestion de bases de datos junto las diferencias entre ellos,
 veremos tambien su historia y cual se adapta mejor a cada tipo de proyecto segun sus puntos fuertes y sus puntos debiles.


\section{MySQL}
\subsection{Historia de MySQL}
En 1995 Michael Widenius junto con su equipo desarrollaron MySQL, siendo lanzada en 2001, este consiste en sistema de almacenamiento
 de archivos orientado a uso domestico y profesional. En 2010 fue comprado por Oracle. Hoy en dia es utilizada por muchos de los sitios web mas importantes, como Facebook, Twitter, Wikipedia, Youtube, Google...
\subsection{Puntos positivos y negativos}
MySQL es sencillo de aprender, bueno para leer datos, compatible con multiples sistemas operativos, cuenta con un buen soporte de la comunidad al ser codigo open-source.\\

Sus puntos debiles son su mal rendimiento en procesos complicados, siendo limitado para actualizar y ampliar la base de datos.

\subsection{Casos de uso}
MySQL es perfecto para proyectos que necesiten leer gran cantidad de datos, no para editado de datos.
 Con su codigo abierto con ayuda de la comunidad, este resulta sencillo de utilizar por lo que es perfecto para proyectos sencillos y complicados.

\section{MongoDB}
\subsection{Historia de MongoDB}

\subsection{Puntos positivos y negativos}

\subsection{Casos de uso}



\section{SQLite}
\subsection{Historia de SQLite}

\subsection{Puntos positivos y negativos}

\subsection{Casos de uso}



\section{PostgreSQL}
\subsection{Historia de PostgreSQL}

\subsection{Puntos positivos y negativos}

\subsection{Casos de uso}



\section{Conclusión}




\end{document}